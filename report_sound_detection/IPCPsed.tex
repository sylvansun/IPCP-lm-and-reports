% the following line is added to eliminate warnings about fonts on Mac
\PassOptionsToPackage{quiet}{fontspec} 

% use ctexarticle and include own package hw-cn.sty
\documentclass[]{ctexart}
\usepackage{hw-cn}

\usepackage{inputenc}
\usepackage{amsfonts,amsmath,amscd,amssymb,amsthm}
\usepackage{latexsym,bm}
\usepackage{cite}
\usepackage{mathtools,mathdots,graphicx,array}
\usepackage{fancyhdr}
\usepackage{lastpage}
\usepackage{color}
\usepackage{enumitem}
\usepackage{diagbox}
\usepackage{xcolor,tcolorbox,tikz,tkz-tab,mdframed,tikz-cd}
\usepackage{framed}
\usepackage{verbatim}
\usepackage{extarrows}
\usepackage{fontspec}
\usepackage{subfigure}
\usepackage{float} % use H command to fix position of figures
\usepackage{graphicx}
\usepackage{listings}
\usepackage{hyperref}
\usepackage{makecell}
\usepackage{bbding}


\newcommand\course{智能感知认知实践}
\newcommand\taskinfo{任务}
\newcommand\name{孙一林}
\newcommand\stuID{520030910361}

\pagestyle{fancyplain}
\headheight 25pt
\lhead{\name \\ \stuID}
\chead{\textbf{\Large 声音事件检测\taskinfo}}
\rhead{\course \\ \today}

\begin{document}

\section{提交文件简要说明}
本次实验使用超算hpc进行,代码改动集中在\texttt{model.py},并未对原有其他代码框架进行大幅更改。数据预处理根据\texttt{data}目录下的\texttt{prepare\_data.sh}文件进行,
提取的数据保存在该目录下,并未从\texttt{stu168}拷贝原始数据集(可以软链接,但并不需要)。按照原有代码框架脚本,实验结果默认保存在\texttt{experiments/Crnn}目录下,
超算运行日志保存在\texttt{slurm\_logs}文件夹下。
\section{任务理解}

\subsection{声音事件检测模型理解}

\subsection{弱监督情况下进行时间轴预测的难点}

\subsection{基线模型设计的原理}

\subsection{Crnn模型实现}

\newpage
\section{实验}

\subsection{数据预处理}
将脚本cp到个人hpc目录后运行\texttt{data}文件夹下的\texttt{prepare\_data.sh}脚本,将根据\texttt{stu168}用户下的原始数据提取用于本次训练的后续相关数据。
需要使用单线程数据处理脚本,并对librosa版本降级,从而成功提取数据。
\subsection{实验结果和分析}
\texttt{baseline.yaml}中未提供大量的参数供调整,本任务的重点在于Crnn的模型搭建和结构调整,故本部分重点在于总结不同Crnn结构下的实验结果,
并展示模型结构改进对评测指标的影响。本次任务最终提交的\texttt{model.py}即保留了最优情况下的模型参数配置,其脚本则为\texttt{run.sh}及相关的配置文件。

本次实验最开始我的直观想法是直接将Crnn所需要的神经网络模块,包括但不限于fully connect, convolution, lstm, batchnorm等融合在一起,
不对参数进行细致的配置,直接使用relu, sigmoid等激活函数,在成功搭建后的初步实验结如下:
\begin{table}[ht]
    \centering
    \begin{tabular}{|l|cll|}
    \hline
    Metrics        & \multicolumn{1}{c|}{f\_measure} & \multicolumn{1}{l|}{precision}  & recall     \\ \hline
    event\_based   & \multicolumn{1}{c|}{0.00743036} & \multicolumn{1}{l|}{0.00666628} & 0.00990206 \\ \hline
    segment\_based & \multicolumn{1}{c|}{0.196863}   & \multicolumn{1}{l|}{0.310598}   & 0.161337   \\ \hline
    tagging\_based & \multicolumn{1}{c|}{0.449451}   & \multicolumn{1}{l|}{0.568862}   & 0.387041   \\ \hline
    mAP            & \multicolumn{3}{c|}{0.5058306091886279}                                        \\ \hline
    \end{tabular}
    \caption{简单Crnn结构:双层卷积}
    \label{exp1}
\end{table}

初步实验成功之后,进行模型结构调整,增加卷积层和BatchNorm层的个数,使用两种不同的池化方式,结果如下:
\begin{table}[ht]
    \centering
    \begin{tabular}{|l|cll|}
    \hline
    Metrics        & \multicolumn{1}{c|}{f\_measure} & \multicolumn{1}{l|}{precision}  & recall     \\ \hline
    event\_based   & \multicolumn{1}{c|}{0.00831543} & \multicolumn{1}{l|}{0.0116714}  & 0.00656456 \\ \hline
    segment\_based & \multicolumn{1}{c|}{0.226929}   & \multicolumn{1}{l|}{0.42916}    & 0.156146   \\ \hline
    tagging\_based & \multicolumn{1}{c|}{0.553278}   & \multicolumn{1}{l|}{0.630679}   & 0.511583   \\ \hline
    mAP            & \multicolumn{3}{c|}{mAP: 0.5788640257440317}                                        \\ \hline
    \end{tabular}
    \caption{改进Crnn结构:三层卷积、batchnorm、两种pooling}
    \label{exp2}
\end{table}

\newpage
按照slides给出模型结构进行最终改进(此即提交文件中的模型结构),结果如下:
\begin{table}[ht]
    \centering
    \begin{tabular}{|l|cll|}
    \hline
    Metrics        & \multicolumn{1}{c|}{f\_measure} & \multicolumn{1}{l|}{precision}  & recall     \\ \hline
    event\_based   & \multicolumn{1}{c|}{0.0108622}  & \multicolumn{1}{l|}{0.0139943}  & 0.00960896 \\ \hline
    segment\_based & \multicolumn{1}{c|}{0.231898}   & \multicolumn{1}{l|}{0.418706}   & 0.163764   \\ \hline
    tagging\_based & \multicolumn{1}{c|}{0.582452}   & \multicolumn{1}{l|}{0.628041}   & 0.565621   \\ \hline
    mAP            & \multicolumn{3}{c|}{mAP: 0.6098265732798768}                                        \\ \hline
    \end{tabular}
    \caption{最终Crnn结构:模型结构保留在model.py中}
    \label{exp3}
\end{table}

模型结构确定,继续进行参数调整,包括但不限于各个卷积层的in\_channels, out\_channels, LSTM网络的隐层维度、层数(即提交文件中的默认参数),
主要是增大卷积层channel的个数和LSTM网络隐层向量的维度,结果如下:
\begin{table}[ht]
    \centering
    \begin{tabular}{|l|cll|}
    \hline
    Metrics        & \multicolumn{1}{c|}{f\_measure} & \multicolumn{1}{l|}{precision}  & recall     \\ \hline
    event\_based   & \multicolumn{1}{c|}{0.00938473} & \multicolumn{1}{l|}{0.0115726}  & 0.00811011 \\ \hline
    segment\_based & \multicolumn{1}{c|}{0.240401}   & \multicolumn{1}{l|}{0.423576}   & 0.172586   \\ \hline
    tagging\_based & \multicolumn{1}{c|}{0.633284}   & \multicolumn{1}{l|}{0.637923}   & 0.634307   \\ \hline
    mAP            & \multicolumn{3}{c|}{mAP: 0.6330678570776226}                                        \\ \hline
    \end{tabular}
    \caption{最终Crnn参数:模型参数是model.py中的默认参数}
    \label{exp4}
\end{table}

\end{document}